%
% API Documentation for API Documentation
% Module CsTransform.pynufft
%
% Generated by epydoc 3.0.1
% [Fri Feb 27 14:49:52 2015]
%

%%%%%%%%%%%%%%%%%%%%%%%%%%%%%%%%%%%%%%%%%%%%%%%%%%%%%%%%%%%%%%%%%%%%%%%%%%%
%%                          Module Description                           %%
%%%%%%%%%%%%%%%%%%%%%%%%%%%%%%%%%%%%%%%%%%%%%%%%%%%%%%%%%%%%%%%%%%%%%%%%%%%

    \index{CsTransform \textit{(package)}!CsTransform.pynufft \textit{(module)}|(}
\section{Module CsTransform.pynufft}

    \label{CsTransform:pynufft}
package docstring author: Jyh-Miin Lin  (Jimmy), Cambridge University 
address: jyhmiinlin at gmail.com Created on 2013/1/21

================================================================================
This file is part of pynufft.

pynufft is free software: you can redistribute it and/or modify it under 
the terms of the GNU General Public License as published by the Free 
Software Foundation, either version 3 of the License, or (at your option) 
any later version.

pynufft is distributed in the hope that it will be useful, but WITHOUT ANY 
WARRANTY; without even the implied warranty of MERCHANTABILITY or FITNESS 
FOR A PARTICULAR PURPOSE.  See the GNU General Public License for more 
details.

You should have received a copy of the GNU General Public License along 
with pynufft.  If not, see 
{\textless}http://www.gnu.org/licenses/{\textgreater}. 
================================================================================

First, see test\_1D(),test\_2D(), test\_3D(), examples


%%%%%%%%%%%%%%%%%%%%%%%%%%%%%%%%%%%%%%%%%%%%%%%%%%%%%%%%%%%%%%%%%%%%%%%%%%%
%%                               Functions                               %%
%%%%%%%%%%%%%%%%%%%%%%%%%%%%%%%%%%%%%%%%%%%%%%%%%%%%%%%%%%%%%%%%%%%%%%%%%%%

  \subsection{Functions}

    \label{CsTransform:pynufft:DFT_slow}
    \index{CsTransform \textit{(package)}!CsTransform.pynufft \textit{(module)}!CsTransform.pynufft.DFT\_slow \textit{(function)}}

    \vspace{0.5ex}

\hspace{.8\funcindent}\begin{boxedminipage}{\funcwidth}

    \raggedright \textbf{DFT\_slow}(\textit{x})

    \vspace{-1.5ex}

    \rule{\textwidth}{0.5\fboxrule}
\setlength{\parskip}{2ex}
    Compute the discrete Fourier Transform of the 1D array x 
    https://jakevdp.github.io/blog/2013/08/28/understanding-the-fft/

\setlength{\parskip}{1ex}
    \end{boxedminipage}

    \label{CsTransform:pynufft:DFT_point}
    \index{CsTransform \textit{(package)}!CsTransform.pynufft \textit{(module)}!CsTransform.pynufft.DFT\_point \textit{(function)}}

    \vspace{0.5ex}

\hspace{.8\funcindent}\begin{boxedminipage}{\funcwidth}

    \raggedright \textbf{DFT\_point}(\textit{x}, \textit{k})

    \vspace{-1.5ex}

    \rule{\textwidth}{0.5\fboxrule}
\setlength{\parskip}{2ex}
    Compute the discrete Fourier Transform of the 1D array x 
    https://jakevdp.github.io/blog/2013/08/28/understanding-the-fft/

\setlength{\parskip}{1ex}
    \end{boxedminipage}

    \label{CsTransform:pynufft:show_3D}
    \index{CsTransform \textit{(package)}!CsTransform.pynufft \textit{(module)}!CsTransform.pynufft.show\_3D \textit{(function)}}

    \vspace{0.5ex}

\hspace{.8\funcindent}\begin{boxedminipage}{\funcwidth}

    \raggedright \textbf{show\_3D}()

\setlength{\parskip}{2ex}
\setlength{\parskip}{1ex}
    \end{boxedminipage}

    \label{CsTransform:pynufft:test_3D}
    \index{CsTransform \textit{(package)}!CsTransform.pynufft \textit{(module)}!CsTransform.pynufft.test\_3D \textit{(function)}}

    \vspace{0.5ex}

\hspace{.8\funcindent}\begin{boxedminipage}{\funcwidth}

    \raggedright \textbf{test\_3D}()

\setlength{\parskip}{2ex}
\setlength{\parskip}{1ex}
    \end{boxedminipage}

    \label{CsTransform:pynufft:test_prolate}
    \index{CsTransform \textit{(package)}!CsTransform.pynufft \textit{(module)}!CsTransform.pynufft.test\_prolate \textit{(function)}}

    \vspace{0.5ex}

\hspace{.8\funcindent}\begin{boxedminipage}{\funcwidth}

    \raggedright \textbf{test\_prolate}()

\setlength{\parskip}{2ex}
\setlength{\parskip}{1ex}
    \end{boxedminipage}

    \label{CsTransform:pynufft:test_2D}
    \index{CsTransform \textit{(package)}!CsTransform.pynufft \textit{(module)}!CsTransform.pynufft.test\_2D \textit{(function)}}

    \vspace{0.5ex}

\hspace{.8\funcindent}\begin{boxedminipage}{\funcwidth}

    \raggedright \textbf{test\_2D}()

\setlength{\parskip}{2ex}
\setlength{\parskip}{1ex}
    \end{boxedminipage}

    \label{CsTransform:pynufft:rFOV_2D}
    \index{CsTransform \textit{(package)}!CsTransform.pynufft \textit{(module)}!CsTransform.pynufft.rFOV\_2D \textit{(function)}}

    \vspace{0.5ex}

\hspace{.8\funcindent}\begin{boxedminipage}{\funcwidth}

    \raggedright \textbf{rFOV\_2D}()

\setlength{\parskip}{2ex}
\setlength{\parskip}{1ex}
    \end{boxedminipage}

    \label{CsTransform:pynufft:test_radial}
    \index{CsTransform \textit{(package)}!CsTransform.pynufft \textit{(module)}!CsTransform.pynufft.test\_radial \textit{(function)}}

    \vspace{0.5ex}

\hspace{.8\funcindent}\begin{boxedminipage}{\funcwidth}

    \raggedright \textbf{test\_radial}()

\setlength{\parskip}{2ex}
\setlength{\parskip}{1ex}
    \end{boxedminipage}

    \label{CsTransform:pynufft:test_1D}
    \index{CsTransform \textit{(package)}!CsTransform.pynufft \textit{(module)}!CsTransform.pynufft.test\_1D \textit{(function)}}

    \vspace{0.5ex}

\hspace{.8\funcindent}\begin{boxedminipage}{\funcwidth}

    \raggedright \textbf{test\_1D}()

\setlength{\parskip}{2ex}
\setlength{\parskip}{1ex}
    \end{boxedminipage}

    \label{CsTransform:pynufft:test_2D_multiprocessing}
    \index{CsTransform \textit{(package)}!CsTransform.pynufft \textit{(module)}!CsTransform.pynufft.test\_2D\_multiprocessing \textit{(function)}}

    \vspace{0.5ex}

\hspace{.8\funcindent}\begin{boxedminipage}{\funcwidth}

    \raggedright \textbf{test\_2D\_multiprocessing}()

\setlength{\parskip}{2ex}
\setlength{\parskip}{1ex}
    \end{boxedminipage}

    \label{CsTransform:pynufft:test_wavelet}
    \index{CsTransform \textit{(package)}!CsTransform.pynufft \textit{(module)}!CsTransform.pynufft.test\_wavelet \textit{(function)}}

    \vspace{0.5ex}

\hspace{.8\funcindent}\begin{boxedminipage}{\funcwidth}

    \raggedright \textbf{test\_wavelet}()

\setlength{\parskip}{2ex}
\setlength{\parskip}{1ex}
    \end{boxedminipage}

    \label{CsTransform:pynufft:histeq}
    \index{CsTransform \textit{(package)}!CsTransform.pynufft \textit{(module)}!CsTransform.pynufft.histeq \textit{(function)}}

    \vspace{0.5ex}

\hspace{.8\funcindent}\begin{boxedminipage}{\funcwidth}

    \raggedright \textbf{histeq}(\textit{im}, \textit{nbr\_bins}={\tt 256})

    \vspace{-1.5ex}

    \rule{\textwidth}{0.5\fboxrule}
\setlength{\parskip}{2ex}
    Histogram equalization of a grayscale image.

\setlength{\parskip}{1ex}
    \end{boxedminipage}

    \label{CsTransform:pynufft:test_SR}
    \index{CsTransform \textit{(package)}!CsTransform.pynufft \textit{(module)}!CsTransform.pynufft.test\_SR \textit{(function)}}

    \vspace{0.5ex}

\hspace{.8\funcindent}\begin{boxedminipage}{\funcwidth}

    \raggedright \textbf{test\_SR}()

\setlength{\parskip}{2ex}
\setlength{\parskip}{1ex}
    \end{boxedminipage}


%%%%%%%%%%%%%%%%%%%%%%%%%%%%%%%%%%%%%%%%%%%%%%%%%%%%%%%%%%%%%%%%%%%%%%%%%%%
%%                               Variables                               %%
%%%%%%%%%%%%%%%%%%%%%%%%%%%%%%%%%%%%%%%%%%%%%%%%%%%%%%%%%%%%%%%%%%%%%%%%%%%

  \subsection{Variables}

    \vspace{-1cm}
\hspace{\varindent}\begin{longtable}{|p{\varnamewidth}|p{\vardescrwidth}|l}
\cline{1-2}
\cline{1-2} \centering \textbf{Name} & \centering \textbf{Description}& \\
\cline{1-2}
\endhead\cline{1-2}\multicolumn{3}{r}{\small\textit{continued on next page}}\\\endfoot\cline{1-2}
\endlastfoot\raggedright c\-m\-a\-p\- & \raggedright \textbf{Value:} 
{\tt matplotlib.cm.gray}&\\
\cline{1-2}
\raggedright n\-o\-r\-m\- & \raggedright \textbf{Value:} 
{\tt matplotlib.colors.Normalize(vmin= 0.0, vmax= 1.0)}&\\
\cline{1-2}
\raggedright \_\-\_\-p\-a\-c\-k\-a\-g\-e\-\_\-\_\- & \raggedright \textbf{Value:} 
{\tt \texttt{'}\texttt{CsTransform}\texttt{'}}&\\
\cline{1-2}
\raggedright b\-p\-a\-s\-s\- & \raggedright \textbf{Value:} 
{\tt array([  6.85477291e-13,   6.85293298e-13,   6.84956443e-\texttt{...}}&\\
\cline{1-2}
\raggedright d\-i\-r\-i\-c\-h\-l\-e\-t\- & \raggedright \textbf{Value:} 
{\tt {\textless}scipy.interpolate.interpolate.interp1d object at 0x7f6b3\texttt{...}}&\\
\cline{1-2}
\raggedright t\- & \raggedright \textbf{Value:} 
{\tt array([-49.9875, -49.975 , -49.9625, ...,  49.975 ,  49.9\texttt{...}}&\\
\cline{1-2}
\end{longtable}


%%%%%%%%%%%%%%%%%%%%%%%%%%%%%%%%%%%%%%%%%%%%%%%%%%%%%%%%%%%%%%%%%%%%%%%%%%%
%%                           Class Description                           %%
%%%%%%%%%%%%%%%%%%%%%%%%%%%%%%%%%%%%%%%%%%%%%%%%%%%%%%%%%%%%%%%%%%%%%%%%%%%

    \index{CsTransform \textit{(package)}!CsTransform.pynufft \textit{(module)}!CsTransform.pynufft.pynufft \textit{(class)}|(}
\subsection{Class pynufft}

    \label{CsTransform:pynufft:pynufft}
\begin{tabular}{cccccc}
% Line for CsTransform.nufft.nufft, linespec=[False]
\multicolumn{2}{r}{\settowidth{\BCL}{CsTransform.nufft.nufft}\multirow{2}{\BCL}{CsTransform.nufft.nufft}}
&&
  \\\cline{3-3}
  &&\multicolumn{1}{c|}{}
&&
  \\
&&\multicolumn{2}{l}{\textbf{CsTransform.pynufft.pynufft}}
\end{tabular}


%%%%%%%%%%%%%%%%%%%%%%%%%%%%%%%%%%%%%%%%%%%%%%%%%%%%%%%%%%%%%%%%%%%%%%%%%%%
%%                                Methods                                %%
%%%%%%%%%%%%%%%%%%%%%%%%%%%%%%%%%%%%%%%%%%%%%%%%%%%%%%%%%%%%%%%%%%%%%%%%%%%

  \subsubsection{Methods}

    \vspace{0.5ex}

\hspace{.8\funcindent}\begin{boxedminipage}{\funcwidth}

    \raggedright \textbf{\_\_init\_\_}(\textit{self}, \textit{om}, \textit{Nd}, \textit{Kd}, \textit{Jd}, \textit{n\_shift}={\tt None})

\setlength{\parskip}{2ex}
    constructor of pyNufft

\setlength{\parskip}{1ex}
      Overrides: CsTransform.nufft.nufft.\_\_init\_\_ 	extit{(inherited documentation)}

    \end{boxedminipage}

    \vspace{0.5ex}

\hspace{.8\funcindent}\begin{boxedminipage}{\funcwidth}

    \raggedright \textbf{initialize\_gpu}(\textit{self})

\setlength{\parskip}{2ex}
\setlength{\parskip}{1ex}
      Overrides: CsTransform.nufft.nufft.initialize\_gpu

    \end{boxedminipage}

    \label{CsTransform:pynufft:pynufft:gpu_k_deconv}
    \index{CsTransform \textit{(package)}!CsTransform.pynufft \textit{(module)}!CsTransform.pynufft.pynufft \textit{(class)}!CsTransform.pynufft.pynufft.gpu\_k\_deconv \textit{(method)}}

    \vspace{0.5ex}

\hspace{.8\funcindent}\begin{boxedminipage}{\funcwidth}

    \raggedright \textbf{gpu\_k\_deconv}(\textit{self})

\setlength{\parskip}{2ex}
\setlength{\parskip}{1ex}
    \end{boxedminipage}

    \label{CsTransform:pynufft:pynufft:gpu_k_modulate}
    \index{CsTransform \textit{(package)}!CsTransform.pynufft \textit{(module)}!CsTransform.pynufft.pynufft \textit{(class)}!CsTransform.pynufft.pynufft.gpu\_k\_modulate \textit{(method)}}

    \vspace{0.5ex}

\hspace{.8\funcindent}\begin{boxedminipage}{\funcwidth}

    \raggedright \textbf{gpu\_k\_modulate}(\textit{self})

\setlength{\parskip}{2ex}
\setlength{\parskip}{1ex}
    \end{boxedminipage}

    \label{CsTransform:pynufft:pynufft:gpu_Nd2KdWKd2Nd}
    \index{CsTransform \textit{(package)}!CsTransform.pynufft \textit{(module)}!CsTransform.pynufft.pynufft \textit{(class)}!CsTransform.pynufft.pynufft.gpu\_Nd2KdWKd2Nd \textit{(method)}}

    \vspace{0.5ex}

\hspace{.8\funcindent}\begin{boxedminipage}{\funcwidth}

    \raggedright \textbf{gpu\_Nd2KdWKd2Nd}(\textit{self}, \textit{x}, \textit{weight\_flag})

    \vspace{-1.5ex}

    \rule{\textwidth}{0.5\fboxrule}
\setlength{\parskip}{2ex}
    Now transform Nd grids to Kd grids(not be reshaped)

\setlength{\parskip}{1ex}
    \end{boxedminipage}

    \label{CsTransform:pynufft:pynufft:gpu_forwardbackward}
    \index{CsTransform \textit{(package)}!CsTransform.pynufft \textit{(module)}!CsTransform.pynufft.pynufft \textit{(class)}!CsTransform.pynufft.pynufft.gpu\_forwardbackward \textit{(method)}}

    \vspace{0.5ex}

\hspace{.8\funcindent}\begin{boxedminipage}{\funcwidth}

    \raggedright \textbf{gpu\_forwardbackward}(\textit{self}, \textit{x})

\setlength{\parskip}{2ex}
\setlength{\parskip}{1ex}
    \end{boxedminipage}

    \label{CsTransform:pynufft:pynufft:true_forward}
    \index{CsTransform \textit{(package)}!CsTransform.pynufft \textit{(module)}!CsTransform.pynufft.pynufft \textit{(class)}!CsTransform.pynufft.pynufft.true\_forward \textit{(method)}}

    \vspace{0.5ex}

\hspace{.8\funcindent}\begin{boxedminipage}{\funcwidth}

    \raggedright \textbf{true\_forward}(\textit{self}, \textit{my\_phantom})

    \vspace{-1.5ex}

    \rule{\textwidth}{0.5\fboxrule}
\setlength{\parskip}{2ex}
    compute the exact NUFT without sparse approximation only for simulation

\setlength{\parskip}{1ex}
    \end{boxedminipage}

    \label{CsTransform:pynufft:pynufft:forwardbackward}
    \index{CsTransform \textit{(package)}!CsTransform.pynufft \textit{(module)}!CsTransform.pynufft.pynufft \textit{(class)}!CsTransform.pynufft.pynufft.forwardbackward \textit{(method)}}

    \vspace{0.5ex}

\hspace{.8\funcindent}\begin{boxedminipage}{\funcwidth}

    \raggedright \textbf{forwardbackward}(\textit{self}, \textit{x})

\setlength{\parskip}{2ex}
\setlength{\parskip}{1ex}
    \end{boxedminipage}

    \label{CsTransform:pynufft:pynufft:pseudoinverse2}
    \index{CsTransform \textit{(package)}!CsTransform.pynufft \textit{(module)}!CsTransform.pynufft.pynufft \textit{(class)}!CsTransform.pynufft.pynufft.pseudoinverse2 \textit{(method)}}

    \vspace{0.5ex}

\hspace{.8\funcindent}\begin{boxedminipage}{\funcwidth}

    \raggedright \textbf{pseudoinverse2}(\textit{self}, \textit{data})

    \vspace{-1.5ex}

    \rule{\textwidth}{0.5\fboxrule}
\setlength{\parskip}{2ex}
    density compensation

\setlength{\parskip}{1ex}
    \end{boxedminipage}

    \label{CsTransform:pynufft:pynufft:pseudoinverse3}
    \index{CsTransform \textit{(package)}!CsTransform.pynufft \textit{(module)}!CsTransform.pynufft.pynufft \textit{(class)}!CsTransform.pynufft.pynufft.pseudoinverse3 \textit{(method)}}

    \vspace{0.5ex}

\hspace{.8\funcindent}\begin{boxedminipage}{\funcwidth}

    \raggedright \textbf{pseudoinverse3}(\textit{self}, \textit{data}, \textit{mu}, \textit{LMBD}, \textit{gamma}, \textit{nInner}, \textit{nBreg})

\setlength{\parskip}{2ex}
\setlength{\parskip}{1ex}
    \end{boxedminipage}

    \label{CsTransform:pynufft:pynufft:pseudoinverse}
    \index{CsTransform \textit{(package)}!CsTransform.pynufft \textit{(module)}!CsTransform.pynufft.pynufft \textit{(class)}!CsTransform.pynufft.pynufft.pseudoinverse \textit{(method)}}

    \vspace{0.5ex}

\hspace{.8\funcindent}\begin{boxedminipage}{\funcwidth}

    \raggedright \textbf{pseudoinverse}(\textit{self}, \textit{data}, \textit{mu}, \textit{LMBD}, \textit{gamma}, \textit{nInner}, \textit{nBreg})

\setlength{\parskip}{2ex}
\setlength{\parskip}{1ex}
    \end{boxedminipage}

    \label{CsTransform:pynufft:pynufft:forwardbackward2}
    \index{CsTransform \textit{(package)}!CsTransform.pynufft \textit{(module)}!CsTransform.pynufft.pynufft \textit{(class)}!CsTransform.pynufft.pynufft.forwardbackward2 \textit{(method)}}

    \vspace{0.5ex}

\hspace{.8\funcindent}\begin{boxedminipage}{\funcwidth}

    \raggedright \textbf{forwardbackward2}(\textit{self}, \textit{x})

    \vspace{-1.5ex}

    \rule{\textwidth}{0.5\fboxrule}
\setlength{\parskip}{2ex}
    Update the data-space

\setlength{\parskip}{1ex}
    \end{boxedminipage}

    \label{CsTransform:pynufft:pynufft:maxrowsum}
    \index{CsTransform \textit{(package)}!CsTransform.pynufft \textit{(module)}!CsTransform.pynufft.pynufft \textit{(class)}!CsTransform.pynufft.pynufft.maxrowsum \textit{(method)}}

    \vspace{0.5ex}

\hspace{.8\funcindent}\begin{boxedminipage}{\funcwidth}

    \raggedright \textbf{maxrowsum}(\textit{self})

\setlength{\parskip}{2ex}
\setlength{\parskip}{1ex}
    \end{boxedminipage}

    \label{CsTransform:pynufft:pynufft:backward2}
    \index{CsTransform \textit{(package)}!CsTransform.pynufft \textit{(module)}!CsTransform.pynufft.pynufft \textit{(class)}!CsTransform.pynufft.pynufft.backward2 \textit{(method)}}

    \vspace{0.5ex}

\hspace{.8\funcindent}\begin{boxedminipage}{\funcwidth}

    \raggedright \textbf{backward2}(\textit{self}, \textit{X})

    \vspace{-1.5ex}

    \rule{\textwidth}{0.5\fboxrule}
\setlength{\parskip}{2ex}
    backward2(x): method of class pyNufft

    from [M x Lprod] shaped input, compute its adjoint(conjugate) of 
    Non-uniform Fourier transform

    INPUT: X: ndarray, [M, Lprod] (Lprod=1 in case 1) where M =st['M']

    OUTPUT: x: ndarray, [Nd[0], Nd[1], ... , Kd[dd-1], Lprod]

\setlength{\parskip}{1ex}
    \end{boxedminipage}

    \label{CsTransform:pynufft:pynufft:adjoint}
    \index{CsTransform \textit{(package)}!CsTransform.pynufft \textit{(module)}!CsTransform.pynufft.pynufft \textit{(class)}!CsTransform.pynufft.pynufft.adjoint \textit{(method)}}

    \vspace{0.5ex}

\hspace{.8\funcindent}\begin{boxedminipage}{\funcwidth}

    \raggedright \textbf{adjoint}(\textit{self}, \textit{f})

    \vspace{-1.5ex}

    \rule{\textwidth}{0.5\fboxrule}
\setlength{\parskip}{2ex}
    adjoint operator to calcualte AT*y

\setlength{\parskip}{1ex}
    \end{boxedminipage}

    \label{CsTransform:pynufft:pynufft:adjoint2}
    \index{CsTransform \textit{(package)}!CsTransform.pynufft \textit{(module)}!CsTransform.pynufft.pynufft \textit{(class)}!CsTransform.pynufft.pynufft.adjoint2 \textit{(method)}}

    \vspace{0.5ex}

\hspace{.8\funcindent}\begin{boxedminipage}{\funcwidth}

    \raggedright \textbf{adjoint2}(\textit{self}, \textit{f})

    \vspace{-1.5ex}

    \rule{\textwidth}{0.5\fboxrule}
\setlength{\parskip}{2ex}
    adjoint operator to calcualte AT*y

\setlength{\parskip}{1ex}
    \end{boxedminipage}


\large{\textbf{\textit{Inherited from CsTransform.nufft.nufft}}}

\begin{quote}
Kd2Nd(), Nd2Kd(), backward(), emb\_fftn(), emb\_ifftn(), finalization(), forward(), gpufftn(), gpuifftn(), linear\_phase(), pipe\_density()
\end{quote}
    \index{CsTransform \textit{(package)}!CsTransform.pynufft \textit{(module)}!CsTransform.pynufft.pynufft \textit{(class)}|)}
    \index{CsTransform \textit{(package)}!CsTransform.pynufft \textit{(module)}|)}
